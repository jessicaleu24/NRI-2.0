
%%%%%%%%% PROJECT SUMMARY -- 1 page, third person
% e.g:  "The PI will prove" not "I will prove"

%Below are the pagination, font size, spacing and margin
%instructions for NSF proposals: \\
%
%FastLane does not automatically paginate a proposal.
%Each section of the proposal must be individually
%paginated prior to upload to the system. \\
%
%Use Computer Modern family of fonts at a font size of 11 points or
%larger. A font size of less than 10 points may be used for mathematical
%formulas or equations, figure, table or diagram captions and when
%using a Symbol font to insert Greek letters or special characters.
%The text must still be readable. The use of small type not in compliance with the NSF guidelines
%may be grounds for NSF to return the proposal without review. \\
%
%No more than 6 lines of text within a vertical space of 1 inch. \\
%
%Margins, in all directions, must be at least an inch. \\
%
%
\setcounter{page}{1}
\renewcommand{\thepage}{\arabic{page}}

\invisiblesection{Project Summary}
% This should be a brief statement of the problem you plan to address.
% It should look something like an abstract. 

%The project summary should be a description of the proposed activity suitable
%for publication, no more than one page in length. It should not be
%an abstract of the proposal, but rather a self-contained description of
%the activity that would result if the proposal were funded. The summary
%should be written in the third person and include a statement of objectives
%and methods to be employed. It should be informative to other persons
%working in the same or related fields and understandable to a scientifically
%or technically literate lay reader. \\
%
%The summary must clearly address in separate statements (within the one-page summary):
%the intellectual merit of the proposed activity; and the broader impacts
%resulting from the proposed activity. Proposals that do not separately
%address both criteria within the one-page Project Summary will be returned without
%review. \\
\subsection{Overview}
In factories of the 21st century, production lines are going to be more and more flexible. Moving robots from cages and employing them in flexible production lines is an inevitable trend to allow flexible manufacturing. Humans and robots are expected to be co-workers and co-inhabitants in factories of the future. Hence it is important to ensure that humans and robots do not harm each other. This proposal is concerned with functional issues to ensure safe and efficient interactions between human workers and the next generation intelligent industrial co-robots.
The goal of this project is to establish a set of design principles for robot safe interaction systems (RSIS) which include robust perception algorithms for environment monitoring and control algorithms for safe human robot interactions (HRI). The proposed RSIS will prevent or minimize occurrences of human-robot collision and robot-robot collision. The RSIS, if applied to current co-robots, will significantly increase the operational speed of robots for higher productivity without sacrificing safety. A new light weight robot arm for use in the experimental validation of the developed RSIS is a proposed element of this project. The lightweight nature of the robot arm will minimize the harm to human subjects in the occurrence of human-robot collision even when the robot is operated at high speed.


\subsection{Intellectual Merit}
% This is why your project is interesting and will help further
% knowledge in the field of mathematics. 

%How important is the proposed activity to advancing
%knowledge and understanding within its own field or across different fields?
%How well qualified is the proposer (individual or team) to conduct the project?
%(If appropriate, the reviewer will comment on the quality of prior work.)
%To what extent does the proposed activity suggest and explore creative, original,
%or potentially transformative concepts? How well conceived and organized is the
%proposed activity? Is there sufficient access to resources?  \\

Safety in HRI is the key issue addressed in this proposal. The proposed work includes hardware design and fabrication, software development (i.e. RSIS), theoretical analysis and experiments with humans. New and unique mechanical features will be introduced to the safe lightweight robot. In the RSIS, an integrated method in dealing with HRI is proposed, which includes visual sensing for robust environment monitoring and human detection, offline and online learning for better understanding of the interactive behavior of humans, and a novel controller design which references the social behavior of humans in human-human interactions (HHI). The novel interactive controller will be designed in a multi-agent framework which addresses the efficiency of the robot motion with safety guarantees. By mimicking the way humans interact with each other, the robot will be able to generate natural interactions between itself and human users. The proposed research will contribute to the following theoretical areas: stochastic optimal control, system identification and multi-agent learning. The human behavioral data obtained from simulations and experiments, which concerns with the interactive behavior of humans in the presence of fully automated intelligent robots, will motivate the design of the next generation human-friendly robot as well as the optimization of task allocation in human-robot cooperation.


\subsection{Broader Impacts}
% There are 4 kinds of broader impacts.
% 1. advance discovery and understanding while promoting teaching,
% training and learning
% 2. broaden the participation of underrepresented groups
% 3. disseminated broadly to enhance scientific and technological
% understanding
% 4. benefits of the proposed activity to society
Physical HRI takes place in many robot applications, such as in medical robots (including nursing robots, rehabilitation devices, etc.), educational robots and autonomous cars. The safety of the human users is the key implementation challenge for human-robot systems. The proposed RSIS deals with general design methodologies and software architectures, which can be introduced to various robot systems with various safety constraints and requirements, and therefore can potentially be applied to various robot systems involving HRI.

The project team includes the Principal Investigator (Masayoshi Tomizuka), two graduate student researchers, and several undergraduate researchers. The PI has a strong record of involving female students in research. He has graduated eleven women Ph.D. students, and currently supervises six other women working towards their Ph.D.s. A female graduate student has been identified for this project, who is also the lead author of several prior papers on the safety of human-robot systems. The PI teaches control and mechatronics courses to diverse groups of graduate and undergraduate students. The PI's laboratory receives numerous visitors including underrepresented students from local high schools, international students and researchers and industrial visitors. % such as the robotic and assistive technology for everyday life. %The research results will be disseminated on the lab website. Publication opportunities on top quality conferences and journals will be seek.
