\invisiblesection{Facilities and Other Resources}

\setcounter{page}{1}
%\renewcommand{\thepage}{Facilities, Equipment, and Other Resources - Page \arabic{page} of 1}

\begin{center}
\textbf{\large Facilities, Equipment \& Other Resources}
\end{center}

%	Identify the facilities to be used, as appropriate, indicate their capacities,
%	pertinent capabilities, relative proximity, and extent of availability to the project.
%	Use ``Other" to describe the facilities at any other performance sites listed and at
%	sites for field studies.

% \textbf{Laboratory:}
% 
% \textbf{Clinical:}
% 
% \textbf{Animal:}

PI Tomizuka's Mechanical Systems Control (MSC) Laboratory includes facilities to support theoretical analysis and software development for the safe and efficient robot collaboration system (SERoCS). In particular, the laboratory has the following capabilities:

\begin{enumerate}
\item Computers: Networked PCs running Windows XP/Windows 7/Ubuntu, deep learning workstation with four NVIDIA Titan X Pascal;
\item Control \& data acquisition facilities: National Instrument LabVIEW FPGA Pioneer System, National Instrument LabVIEW RealTime Module, National Instrument compact RIO, Single-Board RIO, National Instrument data acquisition boards, and Simulink Real-Time target computer;
\item Sensors: one PhaseSpace Impulse X2 Motion Capture system, two Microsoft Kinect stereo cameras and two IDS Ensenso N35 stereo cameras;
\item Robots: Two FANUC LR Mate 200\textit{i}D/7L robots with SMC LEHF grippers and ATI mini 45 force / torque sensors, and one FANUC M-16\textit{i}B robot arm.\end{enumerate}
The MSC laboratory has a floor space of about $3,200 ft^2$, which is adequate for performing all planned experiments using mobile robots.
The PIs also have access to 3D printing and Fused Deposition Modeling (FDM) machines for rapid- prototyping through the ME Department and the Center for Information Technology Research in the Interest of Society (CITRIS) at UC Berkeley. The 3D printer in the ME department can manufacture objects up to 10" x 10" x 12" while the 3D printer at CITRIS can print objects in the size of 8" x 8" x 12".
